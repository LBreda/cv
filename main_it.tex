%%%%%%%%%%%%%%%%%
% This is an sample CV template created using altacv.cls
% (v1.1.3, 30 April 2017) written by LianTze Lim (liantze@gmail.com). Now compiles with pdfLaTeX, XeLaTeX and LuaLaTeX.
% 
%% It may be distributed and/or modified under the
%% conditions of the LaTeX Project Public License, either version 1.3
%% of this license or (at your option) any later version.
%% The latest version of this license is in
%%    http://www.latex-project.org/lppl.txt
%% and version 1.3 or later is part of all distributions of LaTeX
%% version 2003/12/01 or later.
%%%%%%%%%%%%%%%%

\PassOptionsToPackage{dvipsnames}{xcolor}
\documentclass[10pt,a4paper]{altacv}


% Change the page layout if you need to
\geometry{left=1cm,right=9cm,marginparwidth=6.8cm,marginparsep=1.2cm,top=1.25cm,bottom=1.25cm,footskip=2\baselineskip}

\setmainfont{Lato}

% Colors
\definecolor{Blue}{HTML}{2c3e50}
\definecolor{Red}{HTML}{812d3e}
\definecolor{SlateGrey}{HTML}{2E2E2E}
\definecolor{LightGrey}{HTML}{666666}
\colorlet{heading}{Blue}
\colorlet{accent}{Red}
\colorlet{emphasis}{SlateGrey}
\colorlet{body}{LightGrey}

% Bullets
\renewcommand{\itemmarker}{{\small\textbullet}}
\renewcommand{\ratingmarker}{\faCircle}

\begin{document}
\name{Lorenzo Breda}
\tagline{Full-Stack Web Developer}
\photo{2.8cm}{photo}
\personalinfo{%
  \email{lorenzo@lbreda.com}
  \phone{+39\ 3396208225}
% \mailaddress{Via, 00100}
  \location{Roma}
  \homepage{www.lbreda.com}
  \linkedin{linkedin.com/in/lbreda}
  \github{github.com/lbreda}
}

\begin{fullwidth}
\makecvheader
\end{fullwidth}

\cvsection[sidebar_it]{Esperienza professionale
}

\cvevent{Full stack web developer, PM}{TwoBeeSolution S.r.l.}{Dicembre 2015 -- In corso}{Via Pietro Bembo 36a, 00168 Roma}
\begin{itemize}
\item Progettazione e sviluppo di una serie di software per la logistica di costruzione e gestione di linee metropolitane. Tali software sono utilizzati per diverse linee metropolitane in tutto il mondo (Roma, Milano, Napoli, Copenaghen, Aarhus, Riyadh, Salonicco, Lima, Oslo).
\item Progettazione e direzione dello sviluppo di un software gestionale per albi professionali e associazioni. Il software si interfaccia con diversi sistemi della Pubblica Amministrazione.
\item Altri progetti minori in Laravel, NodeJS e Nuxt, amministrazione di piccoli sistemi server.
\end{itemize}

\divider

\cvevent{Webmaster, supporto tecnico}{Istituto Don Calabria}{Aprile 2012 -- Dicembre 2015}{Via Giambattista Soria 13, 00167 Roma}
\begin{itemize}
\item Gestione del sito web istituzionale
\item Supporto tecnico interno
\end{itemize}

\cvsection{Formazione}

\cvevent{Diploma di maturità scientifica}{LSS Evangelista Torricelli, Roma}{Settembre 2002 -- Luglio 2007}{}
\begin{itemize}
\item Ho vinto uno stage presso l'Istituto Nazionale di Fisica Nucleare, durante il quale ho studiato e utilizzato il sistema di simulazione Monte Carlo FLUKA.
\end{itemize}

\cvsection{Altre esperienze formative}
Ho studiato Fisica presso Sapienza Università di Roma tra il 2007 e il 2010 con buoni risultati nell'informatica applicata (principalmente C), nella matematica applicata e nella fisica classica.

\smallskip

Sono passato da Fisica a Informatica nel 2010 con buoni risultati nello studio di matematica applicata, architettura degli elaboratori, programmazione a oggetti, database e sistemi operativi. Ho affrontato anche lo studio delle strutture dati e della programmazione funzionale.

\smallskip

Ho lasciato l'università a seguito di un aumento degli impegni lavorativi.

\cvsection{Interessi}
Sono appassionato di sistemi embeded, e sono in grado di progettare applicazioni basate su RaspberryPI e Arduino.

\smallskip

Sono inoltre un buon fotografo hobbista.

\end{document}
