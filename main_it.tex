%%%%%%%%%%%%%%%%%
% This is an sample CV template created using altacv.cls
% (v1.1.3, 30 April 2017) written by LianTze Lim (liantze@gmail.com). Now compiles with pdfLaTeX, XeLaTeX and LuaLaTeX.
% 
%% It may be distributed and/or modified under the
%% conditions of the LaTeX Project Public License, either version 1.3
%% of this license or (at your option) any later version.
%% The latest version of this license is in
%%    http://www.latex-project.org/lppl.txt
%% and version 1.3 or later is part of all distributions of LaTeX
%% version 2003/12/01 or later.
%%%%%%%%%%%%%%%%

\PassOptionsToPackage{dvipsnames}{xcolor}
\documentclass[10pt,a4paper]{altacv}
\usepackage{hyperref}


% Change the page layout if you need to
\geometry{left=1cm,right=9cm,marginparwidth=6.8cm,marginparsep=1.2cm,top=1.25cm,bottom=1.25cm,footskip=2\baselineskip}

\setmainfont{Lato}

% Colors
\definecolor{Blue}{HTML}{2c3e50}
\definecolor{Red}{HTML}{812d3e}
\definecolor{SlateGrey}{HTML}{2E2E2E}
\definecolor{LightGrey}{HTML}{666666}
\colorlet{heading}{Blue}
\colorlet{accent}{Red}
\colorlet{emphasis}{SlateGrey}
\colorlet{body}{LightGrey}

% Bullets
\renewcommand{\itemmarker}{{\small\textbullet}}
\renewcommand{\ratingmarker}{\faCircle}

\begin{document}
\name{Lorenzo Breda}
\tagline{Full-Stack Developer, software designer}
\photo{2.8cm}{photo}
\personalinfo{%
  \email{\href{mailto:lorenzo@lbreda.com}{lorenzo@lbreda.com}}
  \phone{\href{tel:+393396208225}{+39\ 3396208225}}
% \mailaddress{Via, 00100}
  \location{Roma}
  \homepage{\href{https://www.lbreda.com}{www.lbreda.com}}
  \linkedin{\href{https://www.linkedin.com/in/lbreda}{linkedin.com/in/lbreda}}
  \github{\href{https://github.com/lbreda}{github.com/lbreda}}
}

\begin{fullwidth}
\makecvheader
\end{fullwidth}

\cvsection[sidebar_it]{Qualcosa su di me}
Sono uno sviluppatore web full-stack in ambiente php. Progetto e realizzo sistemi con php, HTML5 e con i più moderni framework JavaScript. Mi trovo a mio agio con la configurazione di sistemi Linux, e non ho particolari difficoltà a muovermi in altri stack e ambienti di programmazione: in particolare mi sono trovato spesso a scrivere piccoli software in C per sistemi embeded e le mie prime esperienze nel mondo di NodeJS sono arrivate da piccoli progetti personali. Sono molto curioso e mi piace affrontare e risolvere problemi con la necessaria creatività.

\cvsection{Esperienza professionale}

\cvevent{Developer / Gestione applicativa}{Servizio Provider Internet del Governatorato}{Giugno 2023 -- in corso}{Stato della Città del Vaticano}
\begin{itemize}
\item Gestione degli applicativi in esercizio presso l'ISP vaticano.
\item Gestione (infrastruttura e registrar) dei DNS per il ccTLD \texttt{.va}.
\item Sviluppo di sistemi web-based per gli enti del Governatorato dello Stato della Città del Vaticano.
\item Configurazione (Ansible) dei sistemi di erogazione di servizi web.
\item Intervento occasionale sui sistemi (Ubuntu, RedHat, infrastruttura di rete).
\end{itemize}

\divider

\cvevent{Full stack web developer, PM}{TwoBeeSolution S.r.l.}{Dicembre 2015 -- Maggio 2023}{Via Pietro Bembo 36a, 00168 Roma}
\begin{itemize}
\item Progettazione e sviluppo di una serie di software per la logistica di costruzione e gestione di linee metropolitane. Tali software sono utilizzati per diverse linee metropolitane in tutto il mondo (Roma, Milano, Napoli, Copenaghen, Aarhus, Riyadh, Salonicco, Lima, Oslo).
\item Progettazione e direzione dello sviluppo di un software gestionale per albi professionali e associazioni. Il software si interfaccia con diversi sistemi della Pubblica Amministrazione.
\item Progettazione e direzione dello sviluppo di un sistema di menu elettronico multilingua per ristoranti (\href{https://atavola.menu}{atavola.menu})
\item Progettazione e sviluppo di un software CRM per un istituto scolastico privato, con annessa gestione degli scuolabus (check-in e check-out degli studenti assegnati a ogni fermata, visualizzazione della flotta tramite riferimenti GPS).
\item Altri progetti minori in Laravel, NodeJS e Nuxt, amministrazione di piccoli sistemi server.
\end{itemize}

\divider

\cvevent{Webmaster, supporto tecnico}{Istituto Don Calabria}{Aprile 2012 -- Dicembre 2015}{Via Giambattista Soria 13, 00167 Roma}
\begin{itemize}
\item Gestione del sito web istituzionale
\item Supporto tecnico interno
\end{itemize}

\pagebreak

\cvsection{Esperienza non professionale}
\begin{itemize}
\item MovieDbBot (\href{https://github.com/LBreda/moviedbbot}{github:LBreda/moviedbbot}), un bot per Telegram che dà informazioni tratte da The Movie Database riguardo attori e film
\item BricksetBot (\href{https://github.com/LBreda/bricksetbot}{github:LBreda/bricksetbot}), un bot per Telegram che permette la ricerca di set LEGO su Brickset 
\item COVID19-IT (\href{https://covid19.lbreda.com/}{covid.lbreda.com}), una dashboard per raccogliere i dati sul COVID-19 da fonti istituzionali in maniera automatizzata
\item Letterboxd Client (\href{https://github.com/nuovi-media/laravel-letterboxd}{github:nuovi-media/laravel-letterboxd}), una libreria Laravel per interrogare le API di Letterboxd
\end{itemize}

\cvsection{Formazione}

\cvevent{Diploma di maturità scientifica}{LSS Evangelista Torricelli, Roma}{Settembre 2002 -- Luglio 2007}{}
\begin{itemize}
\item Ho vinto uno stage presso l'Istituto Nazionale di Fisica Nucleare, durante il quale ho studiato e utilizzato il sistema di simulazione Monte Carlo FLUKA.
\end{itemize}

\cvsection{Altre esperienze formative}
Ho studiato Fisica presso Sapienza Università di Roma tra il 2007 e il 2010 con buoni risultati nell'informatica applicata (principalmente C), nella matematica applicata e nella fisica classica.

\smallskip

Sono passato da Fisica a Informatica nel 2010 con buoni risultati nello studio di matematica applicata, architettura degli elaboratori, programmazione a oggetti, database e sistemi operativi. Ho affrontato anche lo studio delle strutture dati e della programmazione funzionale.

\smallskip

Ho lasciato l'università a seguito di un aumento degli impegni lavorativi.

\cvsection{Interessi}
Sono appassionato di sistemi embeded, e sono in grado di progettare applicazioni basate su RaspberryPi e Arduino.

\smallskip

Sono inoltre un buon fotografo hobbista, particolarmente innamorato della fotografia su pellicola 110.

\end{document}
